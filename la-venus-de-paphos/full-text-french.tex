\documentclass[a4paper, 11pt, oneside, polutonikogreek, french]{article}
% Load encoding definitions (after font package)
\usepackage{ebgaramond}
\usepackage[T1]{fontenc}
\usepackage{textalpha}
\usepackage{graphicx}
\usepackage{float}
\graphicspath{ {./} }
\usepackage[figurename=]{caption}
\usepackage{listings}
\usepackage{subcaption}
\lstset{basicstyle=\ttfamily}

% Babel package:
\usepackage{babel}
% With XeTeX/LuaTeX, load fontspec after babel to use Unicode
% fonts for Latin script and LGR for Greek:
\ifdefined\luatexversion \usepackage{fontspec}\fi
\ifdefined\XeTeXrevision \usepackage{fontspec}\fi

% "`Lipsiakos" italic font `cbleipzig`:
\newcommand*{\lishape}{\fontencoding{LGR}\fontfamily{cmr}%
		       \fontshape{li}\selectfont}
\DeclareTextFontCommand{\textli}{\lishape}

\usepackage{booktabs}
\setlength{\emergencystretch}{15pt}
\usepackage{microtype}
\usepackage{fancyhdr}
\begin{document}
\begin{titlepage} % Suppresses headers and footers on the title page
	\centering % Centre everything on the title page
	%\scshape % Use small caps for all text on the title page

	%------------------------------------------------
	%	Title
	%------------------------------------------------
	
	\rule{\textwidth}{1.6pt}\vspace*{-\baselineskip}\vspace*{2pt} % Thick horizontal rule
	\rule{\textwidth}{0.4pt} % Thin horizontal rule
	
	{\scshape\LARGE La Vénus de Paphos \\ et son Temple}
	
	\rule{\textwidth}{0.4pt}\vspace*{-\baselineskip}\vspace{3.2pt} % Thin horizontal rule
	\rule{\textwidth}{1.6pt} % Thick horizontal rule

	%------------------------------------------------
	%	Subtitle
	%------------------------------------------------
	
	\vspace{1\baselineskip}
	
	{\scshape Par \large M. J. D. Guigniaut \\\footnotesize Professeur de Littérature grecque et Maitre de Conférence à l'École préparatoire de l'Académie de Paris, Membre de la Société asiatique \\\scriptsize Dissertation jointe aux Notes du Tome 4 des Œuvres complètes de Tacite, par J.-L. Burnouf, Professeur au Collège royal de France} % Subtitle or further description
	
    %------------------------------------------------
	%	Cover photo
	%------------------------------------------------
	
	
	%------------------------------------------------
	%	Editor(s)
	%------------------------------------------------
    \vspace*{\fill}

	\vspace{1\baselineskip}

	{\footnotesize\scshape Librairie classique de L. Hachette, Élève de l'ancienne École normale \\ Rue Pierre-Sarrazin, N° 12 \\ et chez Treuttel et Würtz, Libraires, Rue de Bourbon, N° 17}
	
	{\scshape{Paris, 1827}}
	
	\vspace{0.5\baselineskip} % Whitespace after the title block

    \scshape Solar Anamnesis Edition  % Publication year
	
	{\scshape\small CC0 1.0 Universel} % Publisher
\end{titlepage}
\setlength{\parskip}{1mm plus1mm minus1mm}
\clearpage
\vspace*{\fill}
\begin{center}
\emph{À Monsieur Frédéric Creuzer, Professeur de Littérature ancienne à l'Université de Heidelberg, Auteur de la Symbolique et de la Mythologie des Peuples anciens.}
\end{center}
\vspace*{\fill}
\clearpage
\section*{Avant-propos}
\paragraph{}
Nous avons voulu faire de cette courte Dissertation un double quoique bien faible hommage de la reconnaissance au Maître habile dont les ingénieuses méthodes et les admirables exemples dirigèrent nos premiers pas dans le vestibule de l'antiquité ; au Savant profond dont les révélations laborieuses et les fécondes inspirations nous en ouvrirent le sanctuaire. Elle sera en même temps un prélude à quelques morceaux du même genre, qui accompagneront les volumes suivants du \emph{Tacite} de M. Burnouf, et à la publication assez prochaine de la seconde livraison des \emph{Religions de l'Antiquité}, d'après M. Creuzer, laquelle comprendra, outre les religions de l'Asie occidentale, les origines et toute la partie nationale et populaire des religions de la Grèce et de l'Italie anciennes.

Tacite, à la veille de cette grande révolution religieuse qui allait élever un seul Dieu et un seul culte sur les débris de tant de dieux et de cultes divers, semble recueillir avec un curieux pressentiment, toutes les fois que l'occasion s'en présente, les titres épars de ces temples antiques que menaçait une ruine commune. Racontant, au début du second livre des Histoires (chap. 2-4), le retour de Titus en Judée, auprès de Vespasien son père, il nous montre ce prince visitant dans l'île de Chypre le temple de la Vénus de Paphos, célèbre par le concours des indigènes et des étrangers.

« Je ferai, dit-il, sur l'origine de ce culte, l'établissement du temple, la forme de la déesse, qui n'est nulle part ainsi représentée, une courte digression. Le fondateur du temple fut, suivant la tradition la plus ancienne, le roi Aérias ; nom que quelques-uns prétendent au contraire être celui de la déesse. Une opinion plus moderne est que le temple fut consacré par Cinyras, au lieu même où aborda Vénus après que la mer l'eut conçue. On ajoute que la science des aruspices et les secrets de cet art y vinrent du dehors, apportés par le Cilicien Tamiras, et qu'il fut réglé que les descendants de ces deux familles présideraient de concert à tous les soins du culte. Bientôt, pour qu'il ne manquât à la maison royale aucune prééminence sur une race étrangère, les nouveaux-venus renoncèrent à la science qu'ils avaient apportée, et le prêtre que l'on consulte est toujours un descendant de Cinyras. Toute victime est reçue, pourvu qu'elle soit mâle. C'est aux entrailles des chevreaux qu'on a le plus de confiance. Il est défendu d'ensanglanter les autels ; des prières et un feu pur sont tout ce qu'on y offre, et, quoiqu'en plein air, jamais la pluie ne les a mouillés. La déesse n'est point représentée sous la figure humaine : c'est un bloc circulaire, qui, s'élevant en cône, diminue graduellement de la base au sommet. La raison de cette forme est ignorée. »

« Après avoir contemplé la richesse du temple, les offrandes des rois, et toutes ces antiquités que la vanité des Grecs fait remonter à des époques inconnues, Titus, etc. »

\bigskip

(\emph{Traduction de M. Burnouf.})
\clearpage
\section*{La Vénus de Paphos et son Temple}
\paragraph{}
Le passage de Tacite que nous venons de citer d'après son éloquent traducteur, est sans contredit le document le plus important qui nous ait été conservé sur la Vénus de Paphos et son temple, si célèbre dans l'antiquité. Toutefois il est loin de nous donner sur les véritables origines de ce culte des renseignements suffisants. Quant à l'histoire et à l'établissement du temple, il laisse peu de chose à désirer, si ce n'est la figure même de cet édifice ; et s'il décrit celle de l'idole singulière qui représentait la déesse, il garde un silence prudent sur la raison de cette forme évidemment symbolique, aussi bien que sur la part qu'il faut assigner aux fables grecques dans les traditions du temple de Paphos.

Tout annonce que cette part dut être considérable, et que les Grecs, selon leur constante habitude, s'étaient plu à mélanger leurs mythes nationaux avec les mythes étrangers, dans cette longue généalogie qu'Apollodore avait sans doute empruntée à ceux de ses compatriotes qui les premiers écrivirent sur l'île de Cypre.\footnote{Apollod. Biblioth. 3, 14, p. 354, Heyn. ; \emph{add.} obss. p. 323 sqq. ; \emph{ejusd.} obss. ad Homer. Iliad. 11., 20, tom. 6., 118. \emph{Confer.} Meursii Cyprus, 1., 1., Op. t. 3., p. 550 sqq., Flor. 1744.} On y voit \emph{Cinyras}, descendant au sixième degré de l'Aurore et de Céphale, fils de Hersé et petit-fils de Cécrops, premier roi de l'Attique. Ici se révèle clairement l'ouvrage des colons Athéniens et autres, fondateurs de plusieurs villes en Cypre, après le sac de Troie. Par un mélange semblable des traditions, le célèbre Evagoras, roi de Salamine, faisait remonter sa famille d'un côté à Teucer, fils de Télamon, de l'autre à Cinyras, du chef d'une de ses filles.\footnote{Pausan. 1., 3. \emph{Cf.} Isocrat. Evagor. Encom. c. 5 sqq., t. 1., \emph{ed.} Coray.} Tithon, fils de l'Aurore, dans la généalogie précitée, et son époux dans toutes les autres, Phaëthon et Astynoüs, qui viennent ensuite et semblent tenir la place de Memnon ou font songer à lui, rendent raison de l'origine éthiopienne, attribuée par Hérodote à l'une des branches de la population de Cypre, d'après les récits des Cypriens eux-mêmes, ici peut-être encore écho des fables grecques.\footnote{Herodot. 7., 90.} Peut-être aussi sous ces noms, comme sous celui d'\emph{Aërias}, premier auteur du temple élevé en l'honneur de Vénus à Paphos, et qui était le plus ancien de tous les temples de l'île suivant la tradition consignée ailleurs par Tacite,\footnote{Annal. 3., 62.} se cachent d'antiques et primitives relations avec l'Égypte, appelée d'abord \emph{Aëria}, selon quelques-uns.\footnote{C'est l'opinion de Creuzer, \emph{Symbolik und Mythologie}, t. 1., p. 341 sq. Ce nom, du reste, était appliqué à des pays fort divers, à Cypre elle-même, à la Crète, à l'île de Thasos, à la Libye, à l'Éthiopie, à la Sicile. Il est donc bien difficile d'en tirer aucune conclusion déterminée relativement à l'origine de ces mythes. \emph{Confer.} A. Gell. 14., 6 ; Isidor. Orig. 9., 2 ; Hesych. \emph{s. v.}} Mais nous pensons que le nom d'\emph{Aërias} ou \emph{Aëria}, donné par d'autres à la déesse même de Paphos, se rapproche naturellement de celui d'\emph{Aoüs}, premier roi du pays, lequel n'est autre que Tithon ou Phaëthon, comme nous l'apprennent des témoignages formels, et signifie \emph{Enfant de l'Aurore}.\footnote{Etymol. Magn. \emph{voc.} Ἀῶος. \emph{Confer.} Heyn. ad Apollodor. \emph{ubi supra} ; Religions de l'antiquité, d'après Creuzer, tom. 1., part. 1., p. 482, 485 sq. --- Un autre rapprochement, sur la portée duquel nous ne voulons rien préjuger ici, nous est suggéré par Hésychius. Ce lexicographe dit qu'en étolien \emph{aëria} signifie \emph{nuage}, ὀμίχλη. Or, en sanscrit \emph{Abhradatta} veut dire \emph{donné par le nuage, enfant du nuage}, et rappelle involontairement \emph{Aphrodite} ou \emph{Aphrodita}, nom grec de Vénus. Dans Homère ἠερίη est employé avec un rapport évident à l'aurore, au crépuscule ; ἠὴρ pour ἀὴρ est la région des nuages, un nuage même. Iliad. 1., 497, 5., 864, 11., 751, etc.} Ainsi Aërias et Tithon, appelé encore \emph{Astræus}, ne font qu'un, et tous ces êtres de lumière viennent en quelque sorte se donner rendez-vous dans \emph{Cinyras} et dans \emph{Adonis}, son fils, honoré en Cypre sous le nom d'\emph{Aô}, comme l'Aurore se confond elle-même avec Vénus-Aëria, enlevant tour à tour Céphale, Tithon, Phaëthon et Adonis, pour les consacrer aussi bien que Cinyras aux soins de son culte.\footnote{Heyne et Creuzer \emph{ubi supra}, et les auteurs qu'ils ont cités, entre autres Pindar. Pythic. 2., 27 sqq., \emph{et ibi} Schol., tom. 2., p. 508, Heyn. Κινύραν ... Ἱερέα κτίλον Ἀφροδίτας.}

Il y a plus : c'est que Cinyras et Adonis pourraient bien à leur tour n'être qu'un seul et même personnage sous deux noms divers, et nous aurions ici un exemple frappant d'un phénomène qui n'est pas rare dans les religions anciennes, où le dieu, objet du culte, figure aussi comme son premier instituteur, comme le premier roi et le premier prêtre tout ensemble. Et d'abord, Adonis avait régné en Cypre ainsi que Cinyras\footnote{Etymolog. M. \emph{loc. laud.}} : Cinyras, d'un autre côté, était originaire de Byblos en Phénicie, ville fameuse par le culte d'Adonis ou Thammuz, divinité toute phénicienne ; il avait bâti un temple à Vénus sur le mont Liban, et suivant la généalogie même qui nous occupe, c'était d'une fille de Pygmalion, roi Phénicien, qu'il avait eu Adonis.\footnote{Strab. 16., p. 755, Casaub. ; Lucian. de Dea Syria, c. 9 ; Apollodor. \emph{l. l.}} Or, celui-ci se nommait aussi \emph{Gingras}, du nom de la flûte de deuil en Phénicie et en Carie ; et le nom de \emph{Cinyras} ou \emph{Cinyra}, déjà si rapproché de \emph{Gingras}, s'appliquait à un instrument de musique en même temps qu'il exprimait le deuil et les larmes.\footnote{Pollux Onomast. 4., 10, 76 ; Hesych. 2., p. 264 sq. \emph{ibique} Alberti. \emph{Cf.} Phot. Lex. p. 123.} Ajoutons que l'amant infortuné de Vénus, Adonis, s'appelait à Lacédémone \emph{Ciris} ou \emph{Cyris}, dénomination que l'on pourrait rapporter à Κύριος, et qui serait alors une traduction exacte du phénicien \emph{Adonaï} ou le Seigneur, mais dont cette étymologie ne rend peut-être pas une raison suffisante.\footnote{Hesych. 2., p. 266, 387, Albert. ; \emph{conf.} Etymol. M. \emph{s. v.} Suivant cette dernière autorité, Κίῤῥις serait un mot proprement cyprien.} Ce qui est plus sûr c'est que, comme Cinyras et ses descendants furent, suivant la tradition, ensevelis dans le temple de Paphos, Vénus elle-même était censée y reposer auprès de son amant Cinyras, ici évidemment identifié avec Adonis.\footnote{Clem. Alex. Protrept. cap. 3, p. 40, Potter. ; Arnobius adv. gent. 6., p. 85, ed. Paris. 1666 ; Schol. in Gregor. Naz. Carm., p. 35. --- On trouvera tous les développements nécessaires sur le culte d'Adonis, sur les rapports de ce dieu avec Cinyras, etc., dans le livre 4, tome 2., actuellement sous presse, des Religions de l'antiquité d'après Creuzer, de qui nous avons emprunté en grande partie les rapprochements précédents.}

Voilà donc qu'à travers le prestige des mythes grecs nous entrevoyons tout à la fois et la véritable origine et le vrai caractère du culte de Paphos. Ce fut un culte de douleur en même temps que de volupté, un culte asiatique et phénicien ; les noms même de Cinyras et d'Adonis en font foi, et tout porte à penser d'ailleurs que, dès une époque assez reculée, les Phéniciens avaient colonisé l'île de Cypre.\footnote{Creuzer, \emph{Symbolik und Mythologie}, t. 2., p. 91 sqq., et le tom. 2., liv. 4, chap. 3, art. 2, des Relig. de l'antiq. \emph{Conf.} Münter, \emph{der Tempel der himmlischen Gœttin zu Paphos}, p. 2 sqq., \emph{et ibi citata} ; Mannert, \emph{Géographie}, 6., 1, p. 555.} Ce furent eux qui bâtirent l'antique \emph{Golgi}, autre nom de racine phénicienne ; qui instituèrent en ce lieu et sans doute aussi dans l'ancienne Paphos, les honneurs de la Vénus-Uranie ou Céleste, longtemps avant que l'Arcadien Agapénor vînt fonder la ville nouvelle de Paphos, au retour du siége de Troie, et la décorer de temples superbes.\footnote{Pausan. 8., 5 ; Raoul-Rochette, Hist. de l'établissem. des colon. grecq., tom. 2., p. 391 sq. \emph{Cf.} Strab. 14., p. 683 ; Münter, ouvr. c., p. 5, contre Étienne de Byzance et ceux qui avec lui veulent gréciser toute l'île de Cypre, dès l'origine.} Cette Vénus était la même divinité que l'on adorait sous les noms divers de Baaltis ou Dioné, d'Astarté, de Sémiramis, etc., à Byblos, à Sidon, à Ascalon, en un mot dans toute la Phénicie et la Syrie, et qui avait des rapports certains avec la Mylitta de Babylone, l'Alilat des Arabes, la Mitra et l'Anaïtis des Perses et des Arméniens.\footnote{\emph{Voy.} les preuves et les développemens dans Creuzer, \emph{Symb. u. Myth.}, 1., p. 730 sqq. ; 2., 24 sqq., 34, 61, 75, 84, 95, etc. ; surtout Relig. de l'antiq. t. 2., l. 4, chap. 3, \emph{passim. Confer.} Münter, \emph{ubi supra}.} C'était la grande déesse de la nature, considérée dans son apparition céleste, et plus ou moins identifiée tantôt avec la lune, tantôt avec la planète de Vénus, l'étoile du matin, l'aurore. Selon le témoignage d'Hérodote, fondé sur celui des Cypriens eux-mêmes, elle fut apportée d'Ascalon en Cypre, et le père de l'histoire la rapproche de la Vénus de Cythère, dont le temple avait été également fondé par des Phéniciens ou Syriens\footnote{Herodot. 1., 105. \emph{Cf.} Pausan. 1., 14, 3., 23.} ; de Cypre ou de Cythère elle passa dans le Péloponnèse et dans le reste de la Grèce, où elle le confondit plus ou moins avec l'antique divinité nationale, Aphrodite.\footnote{Cette vue nous est commune avec O. Müller, dans sa belle et savante histoire des Doriens, en allem., 1., 405 sq.} Les Grecs et les Romains retrouvant dans différentes contrées l'Astarté phénicienne, la nommèrent tantôt Héra ou Junon, c'est-à-dire la maîtresse, la dame par excellence, comme à Carthage et au promontoire de Lacinium, tantôt Aphrodite ou Vénus, comme au mont Éryx en Sicile. Un fait bien remarquable, et qui vient singulièrement à l'appui des origines asiatiques données à la civilisation de l'Étrurie, c'est qu'au rapport de Strabon\footnote{Lib. 5., p. 241 Casaub. \emph{Cf.} Micali, l'Italie av. la dom. des Rom. t. 2., p. 51 de la trad. fr.} Junon ou la grande déesse populaire de ce pays y portait le nom de \emph{Kupra} ou \emph{Cypra}, qui nous ramène involontairement à l'île de Cypre.\footnote{Le nom de \emph{Cypre} est ordinairement dérivé de \emph{Cyprus}, fille ou fils de Cinyras, d'un arbre ou d'une ville ainsi appelés, etc. \emph{Cf.} Meursius, 1., cap. 2. Cette fille ou ce fils de Cinyras n'est sans doute pas autre que la déesse patronne de l'île, Vénus, qui réunissait les deux sexes, ainsi qu'on le verra plus bas. En nous rappelant le nom du cuivre, \emph{æs cyprium, cuprum}, qui, dès les temps homériques, était travaillé en Cypre avec beaucoup d'art, puisque Agamemnon portait une cuirasse faite d'une composition de ce métal et que Cinyras, suivant la tradition, lui avait envoyée (Iliad. 11., 19 sq.), nous ne saurions nous défendre d'un nouveau rapprochement. \emph{Couvera}, dans les religions de l'Inde, est le dieu des métaux, des richesses, une espèce de Fortune ; il réside au sein des montagnes, et il est environné de \emph{Kinnaras}, génies de la musique. Vénus-Cyprus ou Cypra, Vénus-Uranie, la plus ancienne des Parques, la même que la Destinée et la Fortune (Pausan. 1., 19, 7., 26), son époux Vulcain, ses prêtres les Cinyrades, dont le chef porte le nom d'un instrument de musique, seraient-ils des rencontres de pur hasard ? \emph{Voy.} ci-dessus, et Relig. de l'ant., t. 1., p. 248 sq.}

Les premiers rois de cette île, descendants de Cinyras, et pour cette raison appelés \emph{Cinyrades}, exerçaient à la fois les fonctions de la royauté et celles du sacerdoce. Est-il vrai, comme le pense Creuzer,\footnote{\emph{Symbolik}, 1., p. 342 sqq., et le tom. 2., liv. 4 des Relig. de l'antiq.} qu'il n'en ait pas toujours été ainsi, et qu'à une époque antérieure, les \emph{Tamirades}, autre famille sacrée, originaire de Cilicie, fussent exclusivement en possession des fonctions sacerdotales, tandis que le pouvoir royal aurait seul appartenu aux Cinyrades ? c'est ce qu'il ne nous semble pas possible de conclure du texte de Tacite, où il est dit au contraire que Tamiras ayant importé de Cilicie en Cypre la science et l'art des aruspices, les Tamirades, ses descendants, furent associés, sans doute comme devins, au sacerdoce des Cinyrades, qui dans la suite enlevèrent à cette famille étrangère le privilège qu'ils lui avaient formellement reconnu, et concentrèrent en eux seuls le sacerdoce et la divination avec la royauté (\emph{regium genus ... tantum Cinyrades sacerdos consulitur}). Cependant il nous paraît que jadis une colonie cilicienne avait dû jouer un grand rôle dans l'île de Cypre, qui peut-être même devait à la Cilicie une partie de sa population et des institutions différentes de celles que les Phéniciens y firent prévaloir. Nous en trouverions une preuve dans la généalogie rapportée plus haut, et où figure au cinquième degré un personnage nommé \emph{Sandacus}, immédiatement avant Cinyras. Ce Sandacus, pas plus que Cinyras et Adonis, n'est un nom ni un héros grec. On pourrait, il est vrai, le croire Phénicien où Syrien comme ceux-ci ; car Apollodore le fait venir de Syrie en Cilicie, où il fonda la ville de Célendéris, qui le portait sur ses monnaies,\footnote{\emph{Voy.} Pellerin, Recueil, tab. 73 ; et dans Hunter. \emph{Cf.} Eckhel Doctrina Num. Vet., 3., p. 51 sq.} épousa \emph{Pharnace}, fille de Mégessarus,\footnote{Ce nom chaldéen ou assyrien d'origine ne voudrait-il pas dire le \emph{grand sare}, la \emph{grande année}, sens qui rentrerait tout-à-fait dans le caractère sidérico-mythologique de ces religions ?} et donna le jour à Cinyras, qui retourna en Syrie. Mais le passage même de Tacite que nous commentons nous défend d'interpréter en ce sens celui d'Apollodore ; il donne aux Tamirades ciliciens une origine évidemment différente de celle des Cinyrades, une origine étrangère (\emph{peregrinam stirpem}) ; et d'ailleurs une foule d'autres raisons nous engagent à rattacher plutôt Sandacus à l'Hercule \emph{Sandon} de Lydie (le même que le \emph{Sandes} de la Perse), héros solaire, efféminé, asservi à Omphale, c'est-à-dire à la lune, devenue pouvoir mâle. Cette explication nous semble d'autant plus naturelle, que \emph{Pharnace} rappelle \emph{Pharnaces}, nom sous lequel était adoré le dieu-lune dans les religions du Pont.\footnote{Strab. 12., p. 557. \emph{Cf.} Creuzer, 1., 345 sqq. ; 2., 32, 224, 233 ; et Rel. de l'antiq., liv. 4., chap. 3 et 5.} Un rapprochement non moins intéressant, et qui confirme celui que nous avons tenté plus haut entre le nom de Cypre et la déesse étrusque Cupra, c'est que l'un des trois noms de l'Hercule Sabin était \emph{Sancus}\footnote{\emph{Sancus-Semo-Fidius.} Varro de \emph{L. L.}, 4., 10 ; Propert., 4., 10 ; Ovid. Fastor. 6., 213-217 ; Augustin. de Civ. Dei, 18., 19. \emph{Cf.} Creuzer, 2., p. 964 sq., et Relig. de l'ant., t. 2., liv. 5, sect. 2, chap. 5, art. 1. --- Nous ne voudrions pas trop insister sur ce dernier rapprochement, \emph{Sancus} paraissant être le même nom que \emph{Sanctus}, c'est-à-dire le dieu de la foi jurée (\emph{me Dius fidius, me Hercules}), et dérivant comme lui de la racine \emph{sancire}, laquelle au reste se retrouve dans le sanscrit \emph{sandj}, adhérer, être fidèle. Cependant il est remarquable que Sancus soit un Hercule aussi bien que Sandon et Sandacus : ces trois noms n'auraient-ils pas une origine commune ainsi que les trois personnages ?} ?

Ainsi voilà les religions de l'antique Italie rattachées sur deux points importants à celles de l'Asie-Mineure. Un troisième point de contact, c'est l'art de lire l'avenir dans les entrailles des victimes, qui était également en vigueur dans l'Étrurie, en Lydie, en Cilicie et dans l'île de Cypre, comme il l'avait été chez les plus anciens Grecs.\footnote{Creuzer, 2., p. 937, et Relig. de l'antiq. liv. 5., sect. 2, c. 4, art. 2.} Nous serions tentés de chercher le principe et la source première de ces trois grands rapports plutôt dans la Haute-Asie que sur les côtes de Syrie ; et dans les cultes primitifs de l'Assyrie et peut-être de la Perse, peut-être même de l'Inde, plutôt que dans la religion phénicienne. Quoi qu'il en soit de cette hypothèse qui aurait besoin d'un développement plus étendu, nous savons par Hésychius que le grand-prêtre du temple de Paphos se nommait ἀγήτωρ.\footnote{Hesych. \emph{s. voc.}, t. 1., p. 47, Alb.} Il jouissait de si hautes prérogatives que, lorsque Caton prit possession de l'île au nom du peuple romain, il crut dédommager suffisamment Ptolémée, fils de Ptolémée-Aulétès, qui en était roi, en lui faisant déférer ce sacerdoce suprême.\footnote{Plutarch. Cat. min. c. 35, p. 66, t. 7., Coray.} On voit qu'à cette époque, après les longues révolutions qui avaient agité l'île de Cypre, successivement soumise aux Perses et aux Égyptiens, le sacerdoce et la royauté se trouvaient divisés de nouveau. Les victimes immolées par des prêtres-devins, probablement dans une cour extérieure du temple, puisque le sang ne pouvait couler sur l'autel, devaient être mâles ; ce qui nous rappelle cet autre temple de Vénus surnommée \emph{Acræa}, dans la même île, dont l'accès et même la vue étaient défendus aux femmes.\footnote{Ce temple était situé sur un promontoire nommé Olympe. Strab. 14., pag. 682.} Il y avait là sûrement, comme le pense M. Münter,\footnote{Ouvr. c., p. 21.} un motif mystique qu'il ne serait peut-être pas très difficile de pénétrer. C'était sans doute le même qui faisait attacher une foi singulière aux présages tirés des entrailles des chevreaux ou des boucs, symboles favoris des divinités fécondantes.\footnote{Peut-être était-ce la croyance à un pouvoir générateur, unique et se suffisant à lui-même pour engendrer toutes choses, pouvoir ordinairement représenté par la figure de l'Hermaphrodite qui réunit les deux sexes : on trouvera plus loin une Vénus de Cypre ainsi représentée. Il y avait quelque chose de tout-à-fait analogue dans la religion de Mithra, telle qu'on la concevait en Asie-Mineure : « Mithras, est-il dit, haïssait les femmes, et il féconda un rocher, etc. » \emph{Conf.} Creuzer, 1., 775, et Relig. de l'ant., t. 1., p. 371 sqq., coll. 346 sq.} Le principal autel devait être situé dans la cour intérieure du temple (à moins que le temple lui-même ne fût hypæthre), puisque d'après la tradition populaire rapportée par Tacite et par Pline l'ancien,\footnote{Tac. \emph{l. l.} ; Plin. H. N. 2., 96, où il faut probablement lire \emph{aram} au lieu de \emph{aream}.} quoiqu’en plein air, jamais la pluie ne le mouillait. La pureté et la simplicité du culte qui y était offert contrastent d'une manière frappante avec ce que les auteurs nous apprennent des horribles sacrifices consommés dans les autres temples de l'île, notamment à Salamine et à Amathonte, en l'honneur de Jupiter, c'est-à-dire selon toute apparence, du Baal phénicien ou syrien.\footnote{Lactant. Instit. Divin. 1., 21 ; Ovid. Metam. 10., 224 ; Lutat. in Epitom. \emph{Cf.} Athenæus, 4., 14 ; Münter, \emph{l. l.} p. 22 sq.} Münter conjecture qu'à Paphos l'on prédisait encore l'avenir par le vol des oiseaux et surtout des colombes, qui, dédiées à la déesse du lieu, devaient y être prophétiques aussi bien qu'à Dodone, sur l'Éryx et dans tous les sanctuaires de Vénus-Astarté en Syrie et ailleurs. Les poissons lui étaient-ils également consacrés, comme aux déesses Dercéto et Atargatis, c'est une question que nous ne discuterons point ici et qui souffre plus d'une difficulté.\footnote{Münter, \emph{l. l.} p. 25-28. \emph{Cf.} Creuzer, 2., p. 61-85 ; et Relig. de l'ant., t. 2., liv. 4, ch. 3. --- Les colombes dédiées à Cybèle comme à Vénus-Astarté forment un nouveau point de rapport entre ces religions et le culte de Mithra, où ces oiseaux paraissent avoir joué un rôle symbolique très élevé.}

Il nous reste à parler de la structure du temple et de la figure de l'idole qui représentait la déesse : commençons par celle-ci. C'était, suivant la description pittoresque de Tacite, un véritable cône, ou, comme s'exprime Maxime de Tyr,\footnote{Dissertat. 38. \emph{Cf.} Servius ad Æneid. 1., 724. \emph{In modum umbilici, vel, ut quidam volunt, metæ.}} une pyramide blanche. En effet, c'est sous la première de ces formes que nous l'offrent les pierres gravées, et surtout les médailles frappées en Cypre sous les empereurs, depuis Auguste jusqu'à Macrin.\footnote{Eckhel Doct. Num. Vet. tom. 3., Cyprus, p. 84 sqq. ; Mionnet, Description de médailles, t. 3., p. 670 sq. \emph{Cf.} Lenz, \emph{die Gœttin von Paphos auf alten Bildwerken, und Baphomet}, Gotha, 1808, in-4°, avec les deux planches ; Münter, planches de la dissertat. citée ; et la planche additionnelle qui se rapporte à la présente dissertation, dans la collection de portraits pour les œuvres de Tacite, par P. Bouillon.} Les variétés peu importantes que l'on y remarque paraissent devoir être mises sur le compte des artistes, ou proviennent plutôt encore des ornements divers dont l'idole conique était parée, tels que colliers, guirlandes, voiles, etc. Quelle fut la raison de cette forme, si simple et pourtant si bizarre, donnée à la déesse céleste ? On l'a, selon nous, vainement cherchée dans la barbarie des premiers siècles, qui prenaient des pierres brutes ou grossièrement taillées pour images de divinités non moins grossières, images conservées plus tard, dans des siècles plus éclairés, à cause de leur antiquité même et de la consécration du temps.\footnote{Lenz, dissert. cit., p. 2. \emph{Cf.} Brot., Excurs. ad Tac., p. 26, t. 5., ed. Lemaire.} La manière dont Tacite parle de celle de la Vénus de Paphos, qui d'ailleurs n'avait rien de brut ni de grossier, fait soupçonner au contraire un motif d'un ordre supérieur, un motif caché, symbolique et mystique, comme l'insinue plus clairement Philostrate dans la vie d'Apollonius.\footnote{3., 16. Τὸ τῆς Ἀφροδίτης ἔδος ... ξυμβολικῶς ἱδρυμένον.} C'étaient des symboles que les images divines des Orientaux, c'est-à-dire des idées incorporées à des formes sensibles, qui paraissaient en être l'expression nécessaire. Or, la déesse en question était d'origine orientale, et probablement aussi sa singulière statue ; elle était, comme il a été dit plus haut, la même que l'Astarté des Phéniciens et des Carthaginois. Sur les médailles de Sidon, ainsi que sur certaines pierres trouvées dans les ruines de Carthage, on rencontre souvent des figures coniques ou triangulaires, soit isolées, soit accouplées, qui doivent avoir eu trait au culte de Baal et de sa céleste épouse. L'on a même récemment découvert dans ces dernières ruines un cône d'une dimension considérable, qui fut peut-être à Carthage comme à Paphos l'idole d'Astarté-Uranie.\footnote{Hamaker, diatribe philologico-critica monumentorum aliquot punicorum nuper in Africa repertorum interpretationem exhibens, Lugd. Bat., 1822, p. 27 et tab. 1., 1-4. \emph{Cf.} Münter, \emph{l. l.}, p. 11 sq.} Des cônes, quelquefois avec l'inscription Ἀφροδίτῃ, ont été pareillement trouvés dans différentes parties de la Grèce.\footnote{Dodwell's \emph{a tour through Greece}, vol. 1., p. 34 sq.} Tout annonce donc que la forme conique, l'une des trois formes divines par excellence (les deux autres sont la sphère et le cylindre), suivant une autre inscription découverte à Pergame où le culte de la Vénus de Paphos était florissant,\footnote{\emph{Voy.} cette inscription \emph{isopsèphe}, extrêmement intéressante sous le rapport des connaissances mathématiques des anciens, mais qui n'a trait au sujet qui nous occupe que fort indirectement, dans le Voyage pittoresque de la Grèce de Choiseul-Gouffier, 2., p. 171.} fut sinon exclusivement, au moins principalement consacrée à cette grande déesse. Pourquoi cela ? sans doute encore pour la même raison qui lui faisait dévouer de préférence en victimes les mâles des animaux, et parmi ceux-ci les jeunes boucs, comme au principe divin de toute génération et de toute fécondité. En effet, nous sommes portés à penser avec les interprètes des antiquités d'Herculanum qui ont expliqué en ce sens une des plus remarquables peintures découvertes dans les fouilles de cette ville, que le cône était un symbole de l'amour, et qu'il avait trait au culte si ancien et si profondément significatif du Phallus.\footnote{\emph{Voy. Pitture d'Ercolano}, t. 3., tab. 52., p. 275.} Peut-être même la statue de la déesse de Paphos fit-elle primitivement une espèce de Lingam ou d'Yoni-lingam, pour parler le langage de l'Inde, source probable de ce culte naïf des forces productrices de la nature.\footnote{\emph{Voy.} Relig. de l'antiquité, tom. 1., liv. 1., chap. 2, et les planches qui s'y rapportent dans le tom. 4., section 1.} Peut-être voulut-elle représenter ces forces active et passive dans leur opération commune, par l'emblème des organes sexuels dans leur union,\footnote{Apul. Metamorph. 11., p. 754, Oudend. \emph{Tu, Cœlestis Venus, quæ primis rerum exordiis sexuum diversitatem generato amore sociasti, et æterna sobole humano genere propagato, nunc circumfluo Paphi sacrario coleris.}} comme cette autre idole de Vénus que l'on voyait à Amathonte, succursale connue de Paphos, et qui peignait la déesse sous l'image d'une femme avec une barbe, un sceptre et les caractères non équivoques de l'Hermaphrodite.\footnote{Macrob. Saturn. 3., 8 ; \emph{Cf.} Philochori fragm. p. 19 sq., \emph{ed.} Siebelis. --- Si, comme il paraît, Mithra réunissait en soi les deux sexes et les deux pouvoirs, l'amour, sens de son nom, et le feu constituant du reste son essence, on trouvera vraisemblables les rapprochements de M. Creuzer, d'où il résulte que ce dieu peut avoir été représenté, aussi bien que la Vénus de Paphos, sous la figure d'une colonne en pointe, d'un véritable cône, forme épurée du Phallus et symbole du feu générateur mâle et femelle. \emph{Voy.} Relig. de l'antiq., tom. 1., liv. 2, chap. 4 et 5, \emph{passim}, surtout p. 372 sqq.} Ce qui donne un certain poids à cette conjecture, c'est que le Phallus, au rapport de Clément d'Alexandrie et d'Arnobe, jouait un rôle important dans les initiations secrètes du temple de Paphos.\footnote{Clem. Alex. Protreptr. c. 2, p. 13, Potter ; Arnob. adv. g., 5., p. 74, ed. Paris, 1666. \emph{Cf.} Lucian. de Dea Syria, c. 16.} Münter cherche en vain à se persuader que les images portatives, distribuées aux initiés, dont il est ici question, étaient non des Phallus, mais de petites idoles de la déesse, semblables à celle qu'un marchand de Naucratis, nommé Hérostrate, porta dans sa patrie, après l'avoir achetée à Paphos, vers l'olympiade 23e, et qui n'avait pas plus d'une palme de haut.\footnote{Polycharmus Naucratites ap. Athenæum, 15., 18.} Le texte des auteurs cités est formel, tous les indices rassemblés jusqu'à présent militent en sa faveur ; et Münter avoue lui-même que ces distributions de Phallus n'étaient pas rares dans les mystères de l'antiquité.\footnote{Münter, \emph{l. l.}, p. 18 sq.} Elles ne faisaient sans doute ici que rappeler l'idée fondamentale du culte de Vénus, et peut-être aussi la forme primitive de la déesse, ce qui rendrait d'autant plus futile l'objection de Münter. On voit au reste, par l'exemple précédent, que dès la première moitié du 7e siècle avant notre ère, des cônes sacrés de petite dimension, tels que l'on en trouve aujourd'hui assez fréquemment,\footnote{M. Münter en cite plusieurs à Copenhague, et nous-mêmes nous sommes assez heureux pour en pouvoir publier un d'autant plus caractéristique qu'il porte deux colombes : \emph{voy.} la planche additionnelle indiquée ci-dessus, n° 10. Nous devons ce monument à la précieuse collection de M. Lajard, dont on attend avec impatience l'ouvrage couronné sur le culte et les mystères de Mithra.} étaient également distribués ou plutôt vendus, à Paphos, même aux étrangers. Ces cônes, copies révérées de l'idole du temple, en représentaient la divinité.

Sur les médailles qui sont aujourd'hui pour nous, avec quelques pierres gravées, les seuls monuments du culte de Paphos, c'est dans le sanctuaire même de son temple que paraît la déesse, sous sa forme conique et plus ou moins pyramidale, tantôt nue, tantôt ornée, à ce qu'il semble, de divers accessoires, presque toujours en manière de couronnement. À ses côtés ou au-devant d'elle, brillent deux flambeaux, signes d'orgies nocturnes, comme le pense Creuzer.\footnote{\emph{Symbolik}, 2., pag. 85. \emph{Cf.} Hetsch, \emph{ap.} Münter, \emph{l. l.}, p. 35 ; et la planche additionnelle, particulièrement n° 2.} Peut-être aussi ces flambeaux éclairaient-ils simplement la Cella, tandis que l'idole reposait dans les ténébreuses profondeurs de l'Adytum, selon la coutume. Il est impossible, au surplus, d'après la nature et l'imperfection des représentations figurées qui nous restent, de se faire une idée satisfaisante, tant de la forme générale de l'édifice sacré que du rapport de ses différentes parties entre elles ; et Tacite aussi bien que les autres écrivains de l'antiquité nous laisse à ce sujet dans une ignorance à peu près complète. L'on ne saurait même déterminer avec probabilité la situation de cet autel miraculeux que les eaux du ciel respectaient, à moins qu'il ne fût placé dans cette enceinte demi-circulaire, espèce de vestibule découvert ou de cour intérieure, que tous les monuments indiquent au-devant de la Cella, et qui semble entourée d'une balustrade à jour en forme de grille.\footnote{C'est ce que pensent MM. Münter et Hetsch, dissertation citée, p. 21 et 34. Lenz, p. 13, croit au contraire après Ernesti (sur Tacite, éd. d'Oberlin, reproduite dans la collection Lemaire, t. 3., p. 142) que le temple était \emph{hypæthre}, c'est-à-dire \emph{sub divo}, ouvert par le haut.} Tantôt, comme sur les médailles d'Auguste,\footnote{\emph{Voy.} planche additionnelle, n° 1.} la Cella paraît seule en arrière avec ces deux grandes colonnes latérales que l'on prend fort arbitrairement pour deux obélisques placés à l'entrée du temple, et qui, dans ce cas, seraient plutôt encore deux mâts portant une guirlande, analogues à ceux qui précédaient les temples égyptiens.\footnote{\emph{Ibid.}, n° 3. Il ne faut pas confondre avec ces espèces de mâts, presque toujours divisés à la partie supérieure, les deux petites pyramides ou obélisques qui semblent tenir ici la place des deux candélabres habituels, mais qui peuvent bien aussi avoir existé pour leur propre compte, comme les deux colonnes de la pierre gravée n° 8.} Tantôt, comme sur une médaille d'argent de Vespasien avec l'inscription ΕΤΟΥΣ ΝΕΟΥ ΙΕΡΟΥ,\footnote{\emph{Ibid.}, n° 4. --- Dans le cercle avec un point au milieu, qui est au-dessus du temple, M. Münter croit reconnaître l'image d'un réservoir d'eaux vives avec une colonne au centre, lequel existerait encore dans les ruines du temple de Paphos, suivant un célèbre voyageur, M. de Hammer. Un semblable réservoir abondant en poissons se trouvait, au rapport de Lucien (de D. S., c. 46), près du temple de la déesse de Syrie à Hiérapolis. Il est malheureusement trop probable que ce prétendu réservoir n'est autre chose que le chiffre Θ, partie intégrante de l'inscription.} c'est un édifice plus complet que l'on voit, la Cella étant accompagnée de deux bas-côtés (figurant peut-être un vestibule), bien plus distincts encore et mieux dessinés sur les monnaies de l'impératrice Julia Domna et de Caracalla, son fils.\footnote{\emph{Ibid.}, n° 5. --- On pourrait croire qu'ici l'idole conique se rapproche de la forme humaine si, en la comparant aux n°s 7 et 9, il ne devenait évident qu'ici encore les modifications se réduisent à de simples accessoires ou ornements.} Ces dernières représentations sont les plus riches de toutes : au-dessus de la Cella, dont la partie supérieure semble percée de trois ouvertures, est un croissant ou une demi-lune, surmontée d'une étoile à huit rayons, figurant sans doute, comme sur les médailles phéniciennes et sur d'autres monuments, l'astre de Vénus.\footnote{C'est une conjecture probable de Münter, p. 33, note, et 38. L'on peut aussi rapprocher de cette étoile une mosaïque octogone trouvée dans les ruines.} Les colombes également dédiées à la déesse se montrent partout, dans la cour ou le vestibule demi-circulaire ; dans le sanctuaire, voltigeant autour de l'idole ici singulièrement modifiée, et sur le toit du temple, aux deux bas-côtés. Les médailles de Pergame et de Sardes, deux villes consacrées à la divinité de Paphos, comme le déclarent leurs inscriptions,\footnote{Planche addit., n°s 6 et 7. Comparez encore les deux pierres gravées, n°s 8 et 9. Sur cette dernière, la représentation du temple de Paphos n'est qu'accessoire ; le sujet principal est Sérapis sur son trône, le sceptre en main, le modius sur sa tête et Cerbère aux trois têtes à ses pieds.} offrent quelques autres variantes du même fond, sans nous éclairer davantage.

Les ruines mêmes de l'antique Paphos seraient pour nous une lumière plus sûre, si elles eussent été moins maltraitées par le temps, ou si nous en avions des descriptions plus détaillées. Mais comme l'avoue un savant architecte,\footnote{M. le professeur Hetsch, à la suite de la dissertation citée de M. Münter, p. 30 sqq., --- et les planches 1, 2, surtout 3, qui y sont jointes.} qui n'a pas craint cependant d'essayer une restauration du temple de Vénus, autant la position topographique de cet édifice peut être déterminée avec exactitude d'après les données des derniers voyageurs,\footnote{De Hammer, \emph{topographische Ansichten gesammelt auf einer Reise in die Levante} (Vienne, 1811), p. 134, 150 et les planches ; Voyage d'Ali-Bey, t. 2., p. 144, et l'atlas, pl. 34.} autant il est difficile de se faire une idée quelconque de sa forme et de sa distribution, sans recourir à d'autres documents que nous venons de trouver presque aussi peu satisfaisants sur ces deux points. Ce temple construit d'abord sur un modèle phénicien ou syrien, peut-être sur celui même du temple d'Ascalon,\footnote{Herodot. 1., 105. \emph{Cf.} Lenz, p. 11 ; Münter, p. 5.} paraît avoir été situé sur une hauteur et sur un sol rocailleux, dans la ville ancienne, à peu de distance de la mer.\footnote{A 10 stades des côtes et à 60 E. de la nouvelle Paphos, suivant Strabon, 14., 683. Près du village ture \emph{Koukla}, l'on trouve des ruines assez considérables que Pocoke conjecturait déjà devoir être celles de Palæ-Paphos. M. de Hammer, qui les a visitées depuis, y a découvert de nombreuses inscriptions à Vénus, ΤΗΙ ΑΦΡΟΔΙΤΗΙ, et même, à ce qu'il croit, des inscriptions phéniciennes à demi effacées. Le nom de \emph{Koukla} devait naturellement faire songer à celui de \emph{Golgos} ou \emph{Golgi} (\emph{Voy.} ci-dessus, p. 4) ; mais était-ce une raison suffisante d'identifier cette ancienne ville de Cypre, située dans la partie orientale de l'île, comme on l'avait pensé jusqu'à présent, et tout au moins mentionnée chez les anciens pour son propre compte (Steph. Byz. \emph{in voc., coll.} Plin. H. N. 5., 31 ; Theocr. Adoniaz., 100 \emph{et ibi} Schol. ; Lycophron v. 589 sqq. et sch. \emph{ibid. Conf.} Meurs. Cypr. 1., 2 ; Mannert, \emph{Géographie}, 6., 1, p. 576 sq.) avec Palæ-Paphos, qui était dans la partie occidentale, et que jamais aucun ancien n'a dit s'être appelée Golgi ? C'est pourtant ce que fait M. Münter p. 5 et 7 ; sans déduire explicitement ses motifs, et se fondant seulement, à ce qu'il nous semble, sur le passage où Pausanias (8., 5) dit qu'avant l'arrivée d'Agapénor, fondateur de Paphos et de son temple, Vénus n'était adorée chez les Cypriens qu'à Golgi. Il aurait dû au moins rapprocher de ce passage cet autre endroit du même auteur (1., 14) où il est insinué que les Paphiens furent les premiers adorateurs de Vénus en Cypre, et surtout le passage encore plus formel de Tacite (Ann. 3., 62), où le temple de Paphos est appelé positivement le plus ancien des temples de l'île. Le silence de Strabon sur Golgi et le nom moderne de Koukla paraîtraient alors de nouvelles présomptions en faveur d'une hypothèse gratuite du reste. La nouvelle Paphos des anciens se nomme aujourd'hui \emph{Baffa.} \emph{Voy.} outre de Hammer et Ali-Bey, sur ces localités, Clarke, \emph{Travels}, 2., 328 sqq., 334.} Il est probable que c'est cet ancien temple, comme l'appelle Strabon,\footnote{Ἱερὸν ἀρχαῖον ... ἱερὰ εὖ κατεσκευασμένα, \emph{ubi supra}.} par opposition avec les temples bien bâtis qui décoraient la nouvelle Paphos, il est probable, disons-nous, que c'est l'édifice de Palæ-Paphos que nous représentent les monuments figurés. Là seulement pouvait résider l'idole antique et sacrée, l'idole toute orientale, décrite avec un curieux étonnement par les auteurs Grecs et Romains. Du reste, peut-être ne sommes-nous pas suffisamment éclairés à ce sujet ; mais la forme du sanctuaire, non moins bizarre au fond que celle de la déesse, tels que nous les voyons tous deux sur les médailles, prête beaucoup de force à notre conjecture. D'ailleurs nous savons par le géographe cité plus haut que tous les ans avait lieu une procession très solennelle de la nouvelle à l'ancienne ville ; et cette procession ne pouvait être célébrée qu'en l'honneur de la divinité de cette dernière, sans doute à la fête de Vénus et Adonis, vers le solstice d'été.\footnote{A la fin de Juin, mois qui dans les temps anciens s'appelait en Cypre \emph{Adonis} : Hieronym. Comment. in Ezechiel., 8. \emph{Cf.} Creuzer, 2., 91 sqq. et Relig. de l'antiq. vol. 2., liv. 4, chap. 3, art. 3 ; Münter, p. 17.} Ainsi Palæ-Paphos était encore, au temps de Strabon, la métropole de ce culte antique en Cypre et en Asie-Mineure. Plus d'une fois le temple fut détruit par les tremblements de terre qui dévastèrent l'île, soit depuis, soit aussi sans doute avant les empereurs\footnote{Senec. Nat. Quæst. 6., 26 ; \emph{id.} Ep. 91 ; Dio Cassius, 54., 23, etc. \emph{Cf.} Meursius, 1., c. 18 ; Münter, p. 9 sqq.} ; mais il paraît avoir été constamment rebâti ou restauré sur le type primitif, quoiqu'avec des embellissements et des agrandissements successifs, comme les monuments en font foi. Depuis l'établissement du christianisme, l'oracle d'abord et ensuite le temple lui-même tombèrent en discrédit,\footnote{Apul. Metam. 4., p. 302 Oudend., etc. Münter, \emph{ibid.}} et la nature consomma bientôt une ruine que l'abandon des hommes avait commencée. Toutefois, telle est la puissance des vieux souvenirs, et surtout des souvenirs religieux ! la mémoire du peuple n'a point entièrement oublié le nom jadis et si longtemps révéré de la déesse de Paphos. Si nous en croyons les voyageurs, l'on parle encore en Cypre d'une rein \emph{Aphroditis} ; l'on montre \emph{Koukla}, l'antique Paphos, où fut son sanctuaire ; et près de là, un lieu dont l'appellation barbare \emph{Yeroschipos} semble aussi avoir conservé la trace du nom grec qui désignait les jardins sacrés de Vénus ἱερὸς κῆπος.\footnote{Voyage d'Ali-Bey, 2., p. 129 ; Münter, p. 4.}
\begin{center}
Fin.
\end{center}
\clearpage

\end{document}
